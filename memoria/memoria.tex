%----------
%	CONFIGURACIÓN DEL DOCUMENTO
%----------
\documentclass[12pt]{report} %fuente a 12pt
% MÁRGENES: 2,5 cm sup. e inf.; 3 cm izdo. y dcho.
\usepackage[
  a4paper,
  vmargin=2.5cm,
  hmargin=3cm
]{geometry}

% INTERLINEADO: Estrecho (6 ptos./interlineado 1,15) o Moderado (6 ptos./interlineado 1,5)
\renewcommand{\baselinestretch}{1.15}
\parskip=6pt

% DEFINICIÓN DE COLORES para portada y listados de código
\usepackage[table]{xcolor}
\definecolor{azulUC3M}{RGB}{0,0,102}
\definecolor{gray97}{gray}{.97}
\definecolor{gray75}{gray}{.75}
\definecolor{gray45}{gray}{.45}

\usepackage[a-1b]{pdfx}
\usepackage{hyperref}

\hypersetup{colorlinks=true,
	linkcolor=black, % enlaces a partes del documento (p.e. índice) en color negro
	urlcolor=blue} % enlaces a recursos fuera del documento en azul

% EXPRESIONES MATEMATICAS
\usepackage{amsmath,amssymb,amsfonts,amsthm}

\usepackage{txfonts}
\usepackage[T1]{fontenc}
\usepackage[utf8]{inputenc}

\usepackage[spanish, es-tabla]{babel}
\usepackage[babel, spanish=spanish]{csquotes}
\AtBeginEnvironment{quote}{\small}

% diseño de PIE DE PÁGINA
\usepackage{fancyhdr}
\pagestyle{fancy}
\fancyhf{}
\renewcommand{\headrulewidth}{0pt}
\rfoot{\thepage}
\fancypagestyle{plain}{\pagestyle{fancy}}

% DISEÑO DE LOS TÍTULOS de las partes del trabajo (capítulos y epígrafes o subcapítulos)
\usepackage{titlesec}
\usepackage{titletoc}
\titleformat{\chapter}[block]
{\large\bfseries\filcenter}
{\thechapter.}
{5pt}
{\MakeUppercase}
{}
\titlespacing{\chapter}{0pt}{0pt}{*3}
\titlecontents{chapter}
[0pt]
{}
{\contentsmargin{0pt}\thecontentslabel.\enspace\uppercase}
{\contentsmargin{0pt}\uppercase}
{\titlerule*[.7pc]{.}\contentspage}

\titleformat{\section}
{\bfseries}
{\thesection.}
{5pt}
{}
\titlecontents{section}
[5pt]
{}
{\contentsmargin{0pt}\thecontentslabel.\enspace}
{\contentsmargin{0pt}}
{\titlerule*[.7pc]{.}\contentspage}

\titleformat{\subsection}
{\normalsize\bfseries}
{\thesubsection.}
{5pt}
{}
\titlecontents{subsection}
[10pt]
{}
{\contentsmargin{0pt}
	\thecontentslabel.\enspace}
{\contentsmargin{0pt}}
{\titlerule*[.7pc]{.}\contentspage}

\usepackage{multirow} % permite combinar celdas
\usepackage{caption} % para personalizar el título de tablas y figuras
\usepackage{floatrow} % utilizamos este paquete y sus macros \ttabbox y \ffigbox para alinear los nombres de tablas y figuras de acuerdo con el estilo definido. Para su uso ver archivo de ejemplo
\usepackage{array} % con este paquete podemos definir en la siguiente línea un nuevo tipo de columna para tablas: ancho personalizado y contenido centrado
\newcolumntype{P}[1]{>{\centering\arraybackslash}p{#1}}
\DeclareCaptionFormat{upper}{#1#2\uppercase{#3}\par}

% Diseño de tabla para ingeniería
\captionsetup[table]{
	format=upper,
	name=TABLA,
	justification=centering,
	labelsep=period,
	width=.75\linewidth,
	labelfont=small,
	font=small,
}

\usepackage{graphicx}
\graphicspath{{imagenes/}} %ruta a la carpeta de imágenes

% Diseño de figuras para ingeniería
\captionsetup[figure]{
	format=hang,
	name=Fig.,
	singlelinecheck=off,
	labelsep=period,
	labelfont=small,
	font=small
}

% NOTAS A PIE DE PÁGINA
\usepackage{chngcntr} %para numeración contínua de las notas al pie
\counterwithout{footnote}{chapter}

% LISTADOS DE CÓDIGO
% soporte y estilo para listados de código. Más información en https://es.wikibooks.org/wiki/Manual_de_LaTeX/Listados_de_código/Listados_con_listings
\usepackage{listings}

% definimos un estilo de listings
\lstdefinestyle{estilo}{ frame=Ltb,
	framerule=0pt,
	aboveskip=0.5cm,
	framextopmargin=3pt,
	framexbottommargin=3pt,
	framexleftmargin=0.4cm,
	framesep=0pt,
	rulesep=.4pt,
	backgroundcolor=\color{gray97},
	rulesepcolor=\color{black},
	%
	basicstyle=\ttfamily\footnotesize,
	keywordstyle=\bfseries,
	stringstyle=\ttfamily,
	showstringspaces = false,
	commentstyle=\color{gray45},
	%
	numbers=left,
	numbersep=15pt,
	numberstyle=\tiny,
	numberfirstline = false,
	breaklines=true,
	xleftmargin=\parindent
}

\captionsetup[lstlisting]{font=small, labelsep=period}
% fijamos el estilo a utilizar
\lstset{style=estilo}
\renewcommand{\lstlistingname}{\uppercase{Código}}


%-------------
%	DOCUMENTO
%-------------

\begin{document}
\pagenumbering{roman} % Se utilizan cifras romanas en la numeración de las páginas previas al cuerpo del trabajo

%----------
%	PORTADA
%----------
\begin{titlepage}
	\begin{sffamily}
	\color{azulUC3M}
	\begin{center}
		\begin{figure}[H] %incluimos el logotipo de la Universidad
			\makebox[\textwidth][c]{\includegraphics[width=16cm]{Portada_Logo.png}}
		\end{figure}
		\vspace{2.5cm}
		\begin{Large}
			Máster en Ingeniería Informática\\
			2020 - 2021\\
			\vspace{2cm}
			\textsl{Computación de Altas Prestaciones}
			\bigskip

		\end{Large}
		 	{\Huge ``Paralelización de código con OpenMP + MPI''}\\
		 	\vspace*{0.5cm}
	 		\rule{10.5cm}{0.1mm}\\
			\vspace*{0.9cm}
			{\LARGE Carlos Vigil González}\\
            {\LARGE David Gil López}\\
			{\LARGE Daniel Alejandro Rodríguez López}\\
			\vspace*{1cm}
	\end{center}
	\vfill
	\color{black}
	% \includegraphics[width=4.2cm]{imagenes/creativecommons.png}\\ %incluimos el logotipo de creativecommons
	% \emph{[Incluir en el caso del interés en su publicación en el archivo abierto]}\\  % BORRAR ESTA LÍNEA
	% Esta obra se encuentra sujeta a la licencia Creative Commons \textbf{Reconocimiento - No Comercial - Sin Obra Derivada}
	\end{sffamily}
\end{titlepage}

%----------
%	RESUMEN Y PALABRAS CLAVE
%----------

%----------
%	ÍNDICES
%----------

%--
% Índice general
%-
\tableofcontents
\thispagestyle{fancy}

%--
% Índice de figuras. Si no se incluyen, comenta las líneas siguientes
%-
\listoffigures
\thispagestyle{fancy}

%--
% Índice de tablas. Si no se incluyen, comenta las líneas siguientes
%-
\listoftables
\thispagestyle{fancy}

%----------
%	TRABAJO
%----------
\clearpage
\pagenumbering{arabic} % numeración con múmeros arábigos para el resto de la publicación

\chapter{Introducción}

DANI DICE QUE: ESTO DANI

Aquí se habla sobre la práctica, en qué consiste una equalización de histograma, pasos que hemos seguido durante la práctica y se explica brevemente la experimentación
	% COMENZAR A ESCRIBIR EL TRABAJO

\section{Análisis}

CARLOS DICE QUE: ESTO CARLOS

Aquí creo que habría que comentar de qué formas hemos visto que es posible paralelizar el código inicial y que partes hacerlas por OpenMP y otras por MPI

\subsection{Speed up teóricos}

El Amdhal inicial y el otro, el que variaba con la carga de trabajo


\chapter{OpenMP}

Explicar nuestro objetivo con OpenMP y cómo hemos paralelizado cosas

\section{Paralelización}

\section{Experimentación}


\chapter{MPI}

Explicar nuestro objetivo con MPI y cómo hemos paralelizado cosas

\section{Paralelización}

\section{Experimentación}


\chapter{OpenMP + MPI}

Explicar "conclusiones" o yuqse sobre porqué en base a lo anterior hemos decidido hacer el OpenMP+MPI final, quizás no es exáctamente igual que simplemente juntar lo anterior

\section{Paralelización}

\section{Experimentación}

\chapter{Conclusiones}

Buah colega, hemos aprendido tope mazo

\end{document}
